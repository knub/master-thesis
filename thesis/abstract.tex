\makeatletter\@openrightfalse
\chapter*{Abstract}
Twitter, a popular micro-blogging service, allows sharing news quickly and by nearly everybody. Users often express their opinion on current news and report on events they participated in. Tweets belonging to a news topic provide opportunities for journalists, social scientists, and common users. Furthermore, the analysis of these tweets enables new business applications. In this research, we investigate how tweets related to a specific news event can be identified and used for providing benefit to a user. We pursue two main goals: the search of explicitly or implicitly related tweets for a given news article and their presentation and aggregation. 

For the identification of tweets, we develop a family of search strategies that produce keyword-based search queries applicable to the Twitter Search API. For the estimation of the search term and query quality, the strategies make use of a corpus of current articles and corresponding, explicitly linking tweets. To measure  the quality of the developed approaches, we perform an automatic evaluation using an approximatively gold-standard. Additionally, we carried out a user study with more than 30 participants and 1,300 ranking submissions. The evaluation results show that the strategies perform well in terms of relevance and interestingness of found tweets.

Simplifying the handling of the search results, we implemented a tweet presentation and diversification strategy providing the user with an overview of interesting tweets. Additionally, we explore how the knowledge hidden in tweets can be uncovered and represented. For this purpose, we develop strategies that summarize tweets based on topical, spatial, temporal, and sentiment features. Visualizations implemented on top of these results allow gaining new insights about the underlying news story, like shifts in public mood or the development of public interest.

The presented strategies are implemented in a prototypical web application called ``TweNew''. The application provides access to the developed search and recommendation strategies as well as to the aggregation and visualization techniques. Additionally, we discuss the extension of the prototype to a real-time search engine. %This extension becomes possible, since we designed all strategies paying attention for real-time requirements.

\chapter*{Zusammenfassung}
\@openrighttrue\makeatother
Twitter, ein beliebter Mikroblogging-Dienst, erm\"oglicht es Nachrichten schnell und durch jedermann zu verbreiten. Nutzer berichten dabei h\"aufig \"uber Ereignisse an denen sie teilnahmen oder dr\"ucken ihre Meinung \"uber aktuelle Geschehnisse aus. Tweets mit Bezug zu Nachrichtenthemen stellen eine wichtige Wissensquelle f\"ur Journalisten, Soziologen und gew\"ohnliche Nutzer dar. Durch die Analyse dieser Tweets werden au\ss{}erdem zahlreiche Gesch\"aftsanwendungen m\"oglich. In der vorliegenden Arbeit erforschen wir wie nachrichtenbezogene Tweets identifiziert werden k\"onnen und wie Nutzer davon profitieren k\"onnen. Wir verfolgen dabei zwei Ziele: die Suche aller explizit oder implizit nachrichtenbezogenen Tweets f\"ur einen gegebenen Zeitungsartikel sowie die Empfehlung und Aggregation der Suchergebnisse.

Wir entwickeln dazu eine Familie von Suchstrategien, die, basierend auf Schl\"ussel\-worten, Suchanfragen f\"ur die Twitter Search API generieren k\"onnen. Zur Absch\"atzung der Schl\"usselwort- und Anfragequalit\"at nutzen die Strategien einen Korpus von Artikeln und damit explizit verlinkten Tweets. Zur Evaluierung der vorgestellten Ans\"atze nutzen wir einen approximativen Goldstandard. Au\ss{}erdem wurde eine Nutzerstudie mit mehr als 30 Teilnehmern und 1.300 eingegangenen Bewertungen durchgef\"uhrt. Die Ergebnisse belegen, dass die untersuchten Ans\"atze Tweets finden, die sowohl als relevant als auch als  interessant empfunden werden.

Um dem Nutzer eine \"Ubersicht \"uber die gefundenen Tweets bereitzustellen, entwickeln wir eine Empfehlungs- und Diversifizierungsstrategie. Zus\"atzlich erforschen wir, wie verwertbare Informationen aus den Suchergebnissen gewonnen und  dargestellt werden k\"onnen. Zu diesem Zweck entwickeln wir Strategien, die Tweets anhand von zeitlichen, r\"aumlichen, thematisch- und stimmungsbasierenden Merkmalen zusammenfassen. Auf deren Ergebnissen basierende Visualisierungen erm\"oglichen es Nutzern neue Einsichten \"uber das behandelte Thema zu bekommen. So k\"onnen zum Beispiel die Ver\"anderungen der \"offentlichen Meinung erkannt oder die Entwicklungen des \"offentlichen Interesses nachvollzogen werden.

Die vorgestellten Verfahren werden als Prototypen in einer Webapplikation namens ``TweNew'' umgesetzt. Die Anwendung erm\"oglicht den Zugriff auf die entwickelten Such- und Empfehlungsmethoden sowie Aggregationen und Visualisierungen. Au\ss{}erdem er\"ortern wir eine Erweiterung der Applikation zu einer echtzeitf\"ahigen Suchmaschine.